\message{ !name(bit2019_44664_cat1.tex)}\documentclass[12pt,legal]{article}

\usepackage[margin=1in]{geometry}
\usepackage{titlesec}

\titleformat{\section}
{\Large\bfseries}
{}
{0em}
{}[\titlerule]

\titleformat{\subsection}
{\bfseries\small}
{}
{0em}
{}

\begin{document}

\message{ !name(bit2019_44664_cat1.tex) !offset(-3) }


\section{Mount Kenya University}
\subsection{BIT 1102: Introduction to Computer Programming and Algorithm CAT1}
\subsection{Name: Wilfred Githuka Reg.Number: BIT2019/44664}

Question1:(a) Description of 3 control structures that make up a computer program.
\begin{enumerate}
  \item{Sequence - In sequence, program statements are executed in the sequence
      in which they appear in the proram}
  \item{Selection or Descision - It gives a choice such that if an expression
      is related to a specific statement its executed, otherwise it is skipped.}
    \item{Iteration or Looping - A group of statements in a program may have to
        be executed repeatedly until some condition is satisfied. This is known
        as looping.}
      \end{enumerate}
    \end{document}
Question1:(b) Explanation of a good program
    \begin{enumerate}
    \item{Portability: This is the ability of a program or application to run on
        different platforms (OSs) with or without minimal changes}
      \item{Readability: The program should be written in such a way that it
          makes other programmers or users to follow the logic of the program
          without much effort}
      \end{enumerate}
Question1:(c) Algorithim FlowChart and Pseudocode for KK Security Ltd Wages
Calculation.


\message{ !name(bit2019_44664_cat1.tex) !offset(-50) }

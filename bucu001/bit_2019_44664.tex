\documentclass[12pt, a4]{article}

\usepackage[margin=1in]{geometry}
\usepackage{titlesec}

\titleformat{\section}
{\Large\bfseries}
{}
{0em}
{}[\titlerule]

\titleformat{\subsection}
{\bfseries\small}
{}
{0em}
{}

\begin{document}

\section{Mount Kenya University}
\subsection{BUCU 001: Communication Skills CAT ASSIGNMENT}
\subsection{Name: Wilfred Githuka Reg.Number: BIT2019/44664}

\textbf{Question 1. Discussion of  how the library will affect the learning process of a University.}

A library in a university is very crucial and serves 2 main functions. To
support research and also support curriculum development. Libraries exist to be
a place where students and faculty alike can find knowledge to support or
suppliment the teachings. Libraries also provide a central location for
information exchange between different universities which offer similar
programmes.

The existence of a library can affect learning process in the following ways:

\begin{itemize}
  
\item{School Library Enhance Student Achievement: School libraries exist with
    so much knowledge which the learner can explore and improve the students
    performance. The students can add the library as a source of information to
    their learning process thereby getting a deeper understanding of the taught
    concept after class hours.}

\item{Improve the Research Skills of Students: Most students especially
    undergraduate students lack or have poor research skills. Despite this
    course being taught in universities, many rarely go to the libraries
    therefore when researching about a concept, the work shall be poor.
    Libraries help in sharpening the research skills of students by providing
    mateials where students can read more while also understanding what it means
    to conduct research.}

\item{Teachers' Curricula Development: A library will have a direct impact on the
    content that is being taught. Teahers and other faculty can use the library
    to enrich their teaching mateials and curricula while also adding reference
    materials to the same. They could also recomend materials which the students
    shall use to aid them in getting a deeper understanding of concepts.}

\item{Disrupt The Learning Process: While most university libraries strive to
    provide a positive impact, to some extent it may not be so good. This is to
    mean that students may view the library as a source to too much knowledge
    especially if the concepts have not been grasped well. Student may not use
    the library especially if they find the content not well arranged.}

\item{Audio/Visual Equipment: The audio/visual equipment is meant to provide a
    means or delivering content in a more visual manner. This is just a
    suppliment of the content in the books. However, with the arrival of
    the internet, some learners are finding themselves distracted when they use the
    computers provided at the library. They tend to waste alot of time there
    which affects their overall performance in class.}

\end{itemize}
\pagebreak

\textbf{Question 2: Discussion of the listening strategies with reference to a interview set-up}
\begin{itemize}
  
\item{Good Preparation: Good and adequate preparation before an interview is a
    sure guarantee of having a perfect hearing strategy. Sleeping early, and
    eating light are some of the strategies that one can use to ensure that
    during the interview there are no distractions that may hinder listening}

\item{Good Command of Interview Language: Perhaps the most important is having a good
    command of the langauge from the start. If the interview is being conducted
    in a second langauge, the listener or interviewee may have challenges in the
    listening part. Having a good command of the interview langauge is a good strategy.}

\item{Location: For interviews held on the phone or on the internet, a good
    location is the key strategy. One should find a good location that is quiet
    and has good mobile coverage. Its sad when you can barely here the
    interviewer because of a poor mobile signal. Also for internet video calls,
    place your computer next to the wifi-router to get a better video quality.  }

\item{Clear Any Distractions: Put away any cellphones. One innocent buzz can
    shift your attention and annoy your interviewer. If the interviewer notices
    such distractions they will have a bad attitude towards the interviewee
    quite fast.}

\item{Body Langauge:During the interview show the speaker that you get what
    he/she is saying. Show them using non verbal communication such as
    maintaining eye contact, and body posture.Also a simple nod at the key
    points shows that you understand. This will make the listener focus on
    what the interviewer is saying and also is an encouragement to the speaker
    that the message is understood.}

\end{itemize}

\end{document}